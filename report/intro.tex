
Infinite values in Haskell are commonplace, but why care about partial
values? They are still present because of non-terminating programs,
calls to the error function and pattern-match
failures \cite{chasingbot}.

Why is equality important? There are many reasons why you would want
to know why two Haskell expressions are equal. You might have a naive
way of implementing an algorithm, and an efficient way, and you want
to know that those two implementations behave equally. Your functions
may be a familiar algebraic structure: monoid, group, ring, lattice
and so on, and you want to know that your implementation satisfies all
laws, or your functions may not have such a rich structure but still
have a lot of properties, like idempotency: sorting twice is the same
as sorting one, involution: reversing twice is the same as doing
nothing (but only for finite lists!). Functions could also be
homomorphisms over structures:
\hs{length (xs ++ ys)}$\eq$\hs{length xs + length xs} is an example of
a monoid homomorphism between list concatenation and addition. Haskell
is also known for its type classes like Functor, Applicative and
Monad, and all instances of these should satisfy certain equational
laws.

What is equality? In the general context, a binary relation is an
equivalence relation if it is reflexive ($\fa{x} x \eq x$), symmetric
($\faa{x}{y} x \eq y \rightarrow y \eq x$) and transitive
($\faaa{x}{y}{z} x \eq y \wedge y \eq z \rightarrow x \eq z$). In
Haskell, it is indeed so that every data type can have its own
instance of the \hs{Eq} type class, gives a computable function given
two values of the type and determines if they are equal, the result
given as true or false in a \hs{Bool}. In this project we have made
the simplification that two Haskell values are equal if they have
exactly the same structure: the same constructor at every depth. We
also regard the undefined value $\bot$ as a constructor, so
$\hs{x:}\bot$ is not equal to $\hs{x:[]}$. If the concept of $\bot$ is
unfamiliar to you, do not fear: it will be explained in great
detail. The reason for this simplification is that it will coincide
exactly with the equality in first order logic and allows us to use its
substitution.
\note{Congruence relations!}
