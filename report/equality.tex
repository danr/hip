\section{Definitional Equality}

Some properties cannot or need not use induction or some more
sophisticated technique, since they are true by definition. Examples
are properties for fully polymorphic functions such as the definition
of \hs{id} in the SK-calculus, here

\begin{verbatim}
s f g x = f x (g x)
k x y = x
id x = x

prop_skk_id :: Prop (a -> a)
prop_skk_id = s k k =:= id
\end{verbatim}

Then, the generated conjecture is simply

\begin{equation*}
\app{ (\app {\ptr{s}} {\ptr{k}} )
    }{\ptr{k}} = \ptr{id}
\end{equation*}

Another example where this is useful is to prove functor and monad
laws for the environment monad.

\subsection{Extensional Equality and seq}

To prove the previous property we also need to have extensional
equality, postulated with the following axiom

\begin{equation*}
\faa{f}{g} (\fa{x} \app{f}{x} = \app{g}{x}) \rightarrow f = g
\end{equation*}

which identifies function pointers and functions composed with $@$.
One problem with extensional equality in Haskell, is that the presence
of \hs{seq} breaks it. \hs{seq} is a built in function with the
following behavior:

\begin{verbatim}
seq :: a -> b -> b
seq bottom b = bottom
seq a      b = b
\end{verbatim}

Where \hs{a} is any other value than $\bot$. With \hs{seq} it is
possibly to distinguish between these two functions, which otherwise
are observationally equal:

\begin{verbatim}
f = bottom
g = \x -> bottom
\end{verbatim}

Because \hs{seq f ()} evaluates to $\bot$, and \hs{seq g ()} evaluates
to \hs{()}, but on any argument \hs{f} and \hs{g} gets, they both
return $\bot$. Here we also need an extra axiom, which says that
anything applied to $\bot$ is $\bot$:

\begin{equation*}
\fa{x} \app{\bot}{x} = \bot
\end{equation*}

However, \hs{seq} is the only function that can do this, so we will
ignore its presence in Haskell.
\note{This could be added as a switch \hs{--enable-seq}, which removes
  extensional equality}

Furthermore, if we assume we have extensional equality we also have
the property that \hs{Prop (a -> b)} is equivalent to
\hs{a -> Prop b}, by letting the property have an extra argument that
is applied to the left and right hand side of the equality. This has
two benefits, firstly it can trigger other proof methods should \hs{a}
or \hs{b} be concrete types (the former for induction and the latter
for approximation lemma), and secondly it does not need to use the
extensionality axiom introduced above which is costly and confusing
for the theorem provers tested.
\note{Add some support for this claim. For instance, SPASS seems to be
  utterly horrible at this, but Eprover and Vampire are OK}

\subsection{Concrete Concerns}

This only works on non-concrete types because of the way bottoms are
added. One example when this is a problem with is this plausible
definition of boolean or

\begin{verbatim}
False || a = a
True  || _ = True
\end{verbatim}

\note{This blends a bit with future work. One solution is just to add
  a predicate describing the elements of the type. Hopefully I will have time
  to implement this so it need not be in future work. Further, it is
  pretty sloppily written}
Then an extra branch is added that matches everything else that goes
to $\bot$. This is of efficiency reasons: imagine if we had only used
just a few constructors of a data type with hundreds of constructors,
then we do not want to write a new line with all those constructors to
$\bot$. A property that should indeed hold (regardless of presence of
partial values or not) is \hs{x || False == x} for all
\hs{x}. However, for models with another point, say $\top$, we get
that $\top \, \fn{||} \, \fn{False} = \bot$, and the conjecture is
counter satisfiable.

