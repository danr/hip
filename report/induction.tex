\section{Structural induction}

\subsection{Induction}

Some background of induction and how it is present in well-known
theories like PA, ZFC, MLTT and CoC.

PA which only concerns natural numbers has a small vocabulary
consisting only of the constant $0$, the unary successor function $s$,
and binary plus and multiplication.
Here the induction schema from looks like this:

\note{One could also be explicit about the free variables in $P$}
\begin{mathpar}
  \inferrule*
     {
       \overbrace{P(0)}^{\mathrm{base}}
       \\
       \overbrace{
           \fa{x}
                 \underbrace{P(x)}_{\mathrm{hypothesis}}
              \rightarrow
                 \underbrace{P(s(x))}_{\mathrm{conclusion}}
       }^{\mathrm{step}}
     }
     { \fa{x} P(x) }
\end{mathpar}

This is a axioms schema since it is not possible to quantify over the
predicate $P$ in FOL. Rather, it is a infinite set of axioms, one for
each (well-formed) formula instantiated in place for $P$. Generally,
ATPs do not instantiate schemas themselves but it has to be done
manually, with an appropriate formula for $P$.

Any non-recursive, or more importantly recursive data type gives rise
to induction schemata.

\footnote{Haskell's natural numbers are of course also cluttered with
  elements that are not natural numbers, such as $\bot$, but also the
  infinite ``natural number'' defined by \hs{infinite = Succ infinite}}
, defined the usual way in Haskell by \hs{data Nat = Zero | Succ Nat}
yields this induction axiom schema:

\begin{mathpar}
  \inferrule*
     {
       P(\fn{Zero})
       \\
       \fa{x} P(x) \rightarrow P(\fn{Succ}(x))
     }
     { \fa{x} P(x) }
\end{mathpar}
