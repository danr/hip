\section{Structural induction}

Induction is a very fundamental concept of mathematics and is directly
or indirectly an axiom in axiomatizations of mathematics: directly in
Peano Arithmetic (abbreviated PA), indirectly in Zermelo-Fraenkel Set
Theory (commonly abbreviated ZF or ZFC with the Axiom of Choice). In
type theory there are different approaches: the proof assistants Coq
and Isabell/HOL generates an induction schema for each data type, and
in Agda, based on Martin-Löf Type Theory with inductive families,
induction is a matter of recursion.

PA, or first order aritmetic, has the natural numbers as standard
model and small vocabulary consisting only of the constant $0$, the unary
successor function $s$, and binary plus and multiplication.  Here the
induction schema from looks like this:

\note{One could also be explicit about the free variables in $P$}
\begin{mathpar}
  \inferrule* [Left=PA-Ind]
     {
       \overbrace{P(0)}^{\mathrm{base}}
       \\
       \overbrace{
           \fa{x}
                 \underbrace{P(x)}_{\mathrm{hypothesis}}
              \rightarrow
                 \underbrace{P(s(x))}_{\mathrm{conclusion}}
       }^{\mathrm{step}}
     }
     { \fa{x} P(x) }
\end{mathpar}

Different parts of this rule has been highlighted that are common
names for induction proofs: the induction base, the induction step
with the assumed hypothesis and the obligated conclusion.  This is a
axiom schema since it is not possible to quantify over the predicate
$P$ in FOL. Rather, it is a infinite set of axioms, one for each
(well-formed) formula instantiated in place for $P$. Generally, ATPs
do not instantiate schemas themselves but it has to be done manually,
with an appropriate formula for $P$.

\note{
The set of relevant things to say here are probably not even
recursively enumerable: we could go on presenting induction in
different theories and how it encodes or generates axioms for
structural induction.}

To prove a property of the total and finite Peano Natural numbers in
Haskell since the have the same structure as numbers in PA, the
following axiom schema is used:

\begin{mathpar}
  \inferrule*
     {
       P(\hs{Zero})
       \\
       \fa{x} P(x) \rightarrow P(\hs{Succ}(x))
     }
     { \fa{x} x \mathrm{\w finite \w and \w total} \rightarrow P(x) }
\end{mathpar}

If $P$ is a predicate that preserves $\sqsubseteq$-chains, further
discussed in \ref{sec:admissibility}, then if you prove $P(\bot)$,
then $P(x)$ is true for all $x$, as well as partial and infinite. This
is because this is a $\sqsubseteq$-chain:

\begin{equation*}
\bot \sqsubseteq
\hs{Succ} \w \bot \sqsubseteq
\hs{Succ} \w (\hs{Succ} \w \bot) \sqsubseteq
\hs{Succ} \w (\hs{Succ} \w (\hs{Succ} \w \bot)) \sqsubseteq
\cdots
\end{equation}

with limit \hs{inf}, defined as \hs{inf = Succ inf}. If $P$ then
preserves $\sqsubseteq$-chains, the property also holds for
\hs{inf}. Equality properties are admissible.

Any non-recursive, or more importantly recursive data type gives rise
to induction schemata, and this is called structural induction. Here
are the axiom schemata for potentially partial and infinite lists and
the \hs{Tree} type defined in \ref{sec:treetrans}:

\begin{mathpar}
  \inferrule* [left=List-Ind]
     {
       P(\bot)
       \\
       P(\hs{[]})
       \\
       \faa{x}{xs} P(xs) \rightarrow P(x\hs{:}xs)
     }
     { \fa{x} P(x) }

  \inferrule* [left=Tree-Ind]
     {
       P(\bot)
       \\
       P(\hs{Empty})
       \\
       \faaa{l}{x}{r} P(l) \wedge P(r) \rightarrow P(\hs{Branch} \w l \w x \w r)
     }
     { \fa{x} P(x) }
\end{mathpar}

For such simple indexed sum of products data types the translation is
straightforward: for a data type $T$, with a constructor $K$ you get
all recursive arguments of $T$ to $K$ as hypotheses. It is also
possible to get structural induction for exponential data types. An
example is higher-order abstract syntax, another is
\note{Cite, for example, Dixon's thesis}the Brouwer ordinals, defined as:

\begin{verbatim}
data Ord = Zero | Succ Ord | Lim (Nat -> Ord)
\end{verbatim}

\note{This is actually not yet implemented :(}
The induction schema for this data type is:

\begin{mathpar}
  \inferrule* [left=Ord-Ind]
     {
       P(\bot)
       \\
       P(\hs{Zero})
       \\
       \fa{f} (\fa{x} P(f \w x)) \rightarrow P(\hs{Lim} \w f)
     }
     { \fa{x} P(x) }
\end{mathpar}

\subsection{Applications}

The examples where

\subsection{Generalizations}

The structural induction schemata we have seen so far only go one
depth down the data type, and that is of limited use when proving
properties about functions defined with recursion with a bigger
depth. For instance, the induction on Peano numbers in Haskell is
adjusted to depth two in this way:

\begin{mathpar}
  \inferrule* [left=Nat-Ind-Depth-Two]
     {
       P(\bot)
       \and
       P(\hs{Zero})
       \and
       P(\hs{Succ Zero})
       \\
       \fa{x} P(x) \wedge P(\hs{Succ} \w x) \rightarrow P(\hs{Succ} \w (\hs{Succ} \w x))
     }
     { \fa{x} P(x) }
\end{mathpar}

If you want to prove a property that has a list of natural numbers as
property, you can get a very strong induction hypothesis, by unrolling
the data type enough.