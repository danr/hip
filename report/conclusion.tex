\chapter{Conclusion}
\label{ch:conclusion}

In this thesis we developed a tool able to prove equational properties
for Haskell programs. This was accomplished by means of a translation
to first order logic and instantiation of several different induction
techniques for functional programs. The proof search is carried out by
automated theorem provers. The translation handles partial values and
some of the available proof techniques can prove properties about
corecursive programs generating infinite values, and properties that
hold for infinite and partial arguments.

An exhaustive test of the system was carried out on a test suite
developed as a part of this project. The test suite consists of over
500 properties. % presumably true properties
The approach of using automated theorem provers on
this kind of problems turned out to be successful; the proved theorems
often took much less time than 100 milliseconds for the theorem
provers. For the properties in the test suite, over 200 theorems were
proved to hold for infinite and partial values, and for total and
finite values more than 100. There were properties that could be
proved with many techniques, but every technique had some properties
that it could exclusively prove.

The future work includes adding a system of adding proved properties
as lemmas for proving other properties. This turned out to be
difficult for two reasons. First, the translation of pattern matching
needs to be carried out in a certain way. Second, adding lemmas that
only holds for finite values needs to be tagged by means of predicates
or functions in the generated theory that witnesses their
finiteness. To appropriately use these tags, it would be necessary to
prove termination of functions, and conversely explain which arguments
are finite given that a function terminate with a finite value.
To generate suitable lemmas it would be interesting to automatically
synthesise conjectures, and verify the specifications by testing
before proving. Such tools for specifications are QuickSpec
\citep{quickspec} and IsaCosy \citep{isacosy}. This would relief the
user of the burden of figuring out exactly which lemmas are needed to
prove desired properties.

Finally, two of Haskell's beauties is its purity and simplicity, and
the simple and effective technique of equational reasoning is readily
available thanks to referential transparency. We believe that this
tool is on the right track of using these properties to accomplish
automatic verification of Haskell programs.  It should be noted that
the tool is able to prove properties without type information and
without proving termination for functions.  As this thesis shows,
first order logic is expressible enough to capture many different
aspects of Haskell, including closures of higher order functions, even
though no native support for quantifying over functions is available
in the logic. By using automated theorem provers the
burden of reimplementing a proof system is avoided.
