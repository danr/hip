\chapter{Conclusion}
\label{ch:conclusion}

In this thesis we developed an tool able to prove equational
properties for Haskell programs. This was accomplished by means of a
translation to first order logic and instantiation of several
different proving techniques for functional programs. The proof search
is carried out by automated theorem provers. The translation handles
partial values and some proof techniques can prove properties about
corecursive programs generating infinite values, and properties that
hold for infinite or partial arguments.

An exhaustive test of the test suite developed by side consisting of
over 500 such properties was carried out. The approach of using
automated theorem provers on this kind of problems turned out to be
successful; the proved theorems took often took much less time than
100 milliseconds for the theorem provers. The number of proved
theorems for infinite and partial values were over 200, and for total
and finite values more than 100. There were both overlap in the
different proof techniques, and properties that could only be proved
with a certain technique.

The future work includes adding a system of adding proved properties
as lemmas for proving other properties. This turned out to be
difficult of two reasons. First, the translation of pattern matching
needs to be carried out in a certain way. Second, adding lemmas that
only holds for finite values needs to be tagged by means of predicates
or functions in the generated theory that witnesses their
finiteness. To appropriately use these tags, it would be necessary to
prove termination of functions, and conversely explain which arguments
are finite given that a function terminate with a finite value.

To generate lemmas it would be interesting to automatically deduce
specifications via means of testing. Such tools for specifications are
QuickSpec \citep{quickspec} and IsaCosy \citep{isacosy}.
