\section{Related Work}

\note{Related work in background or in discussion? Would be nice to
  have it after the technical part to be able to compare different
  approaches.}

Zeno \cite{zeno} is the most similar work to this, proving equational
properties for Haskell programs. While Zeno supports type classes, and
implications with equational antecedents, and can generalize goals, it
makes no attempt to tackle infinite and partial values. Indeed, the
authors even ask themselves if it is not really a ML theorem prover.

Zeno translates its proofs to Isabel, which could maybe be described
as a framework for different logics: its most prominent use is as a
theorem prover for Higher Order Logic, commonly abbreviated
Isabel/HOL. Different automated proving techniques has been
implemented, most notably the heuristic Rippling \cite{dixonphd},
found in IsaPlanner \cite{isaplanner}, which also relies on critics
\cite{productiveuse} You can also call first order logic theorem
provers from Isabel/HOL with Sledgehammer.

The proof assistant and programming language Coq resembles Isabel in
the sense that each data type declaration generates an induction
principle and the user can write own induction procedures should it be
too weak. Coq has is quite automated in its use of tactics to resolve
goals, and has an own language Ltac to write such tactics.

Coq has some support for corecursion, and so has the dependently
language Agda, whose strong type system based on Martin-Löf type
theory with inductive families, makes it a proof assistant thanks to
the Curry-Howard-Lambek correspondence. Its support for corecursion
has been investigated \cite{productivity}, although all of it is
manual.

Effort has been put into automating the proof generation in Agda,
notably by \cite{agdaatp}, which is quite alike this work in its
translation from Agda to first order logic. One must however give
quite much information to the theorem prover, like how to instantiate
the induction hypotheses. Peter Dybjer maybe said that it could be
seen as a very nice way to interact with a theorem prover, thanks to
Agda. We see a more constructive (in the logic sense) approach in
\cite{dtatp}, where they do not only make calls to the Waldmeister
theorem prover to assert proof's truth, but they also translate the
prover's output back to functional code for their dependently language
Mella, not much unlike Agda. It is still in early stages and has no
support for data types.

These automated approaches to Agda do not handle infinite values. How
to prove properties about corecursive Haskell programs and similar
languages has long been researched, insert suitable fixed point
induction reference, and new approaches to stream processing can be
found in \cite{streams}.
