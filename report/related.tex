\section{Related Work}

Zeno \citep{zeno} is the most similar work to this, proving equational
properties for Haskell programs. While Zeno supports type classes, and
implications with equational antecedents, and can generalize goals, it
makes no attempt to tackle infinite and partial values. Indeed, the
authors even ask themselves if it is not really a ML theorem prover.

%\note{Discuss this with Moa and get references}
Zeno translates its
proofs to Isabelle, which could maybe be described as a framework for
different logics: its most prominent use is as a theorem prover for
Higher Order Logic, commonly abbreviated Isabelle/HOL
\citep{isabelbook}. Different automated proving techniques has been
implemented, most notably the heuristic Rippling \citep{dixonphd},
found in IsaPlanner \citep{isaplanner}, which also relies on critics
\citep{productiveuse}. You can also call first order logic theorem
provers from Isabelle/HOL with Sledgehammer.

The proof assistant and programming language Coq resembles Isabelle in
the sense that each data type declaration generates an induction
principle and the user can write own induction procedures should it be
too weak. Coq has is quite automated in its use of tactics to resolve
goals, and has an own language Ltac to write such tactics, and a
rippling tactic has been implemented \citep{automateddtfp},
\citep{inductioncoq}.

Coq has some support for corecursion, and so has the dependently
language Agda \citep{norell:thesis}, whose strong type system based on
Martin-Löf type theory \citep{mltt} with inductive families, makes it a proof
assistant thanks to the Curry-Howard-Lambek correspondence. Its
support for corecursion has been investigated \citep{productivity},
although all of it is manual.

Effort has been put into automating the proof generation in Agda,
notably by \cite{agdaatp}, which is quite alike this work in its
translation from Agda to first order logic. One must however give
quite much information to the theorem prover, like how to instantiate
the induction hypotheses. %\note{Peter Dybjer maybe said that it could be seen as a very nice way to interact with a theorem prover, thanks to Agda.}
A more constructive approach by \cite{dtatp}, where they
do not only make calls to the Waldmeister theorem prover to assert
proof's truth, but they also translate the prover's output back to
functional code for their dependently language Mella, not much unlike
Agda. It is still in early stages and has no support for data
types. A similar approach was taken by \cite{agsy}
does a search akin to Prolog inspired search, and can also handle
induction. A living incarnation of this is the Agsy proof search
assistant in Agda.

These automated approaches to Agda do not handle infinite values. How
to prove properties about corecursive Haskell programs and similar
languages has long been researched, insert suitable fixed point
induction reference, and new approaches to stream processing can be
found in the work by \cite{streams}.

A radically different way to reason about Haskell programs than this
thesis does is investigated by \cite{fastandloose}. The idea is to do
reasoning in a language assumed to only be total, and under many
circumstances the proofs also holds for partial values.
