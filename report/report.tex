\documentclass{report}
\usepackage[utf8]{inputenc}
\usepackage{amssymb}
\usepackage{graphicx}
\usepackage[margin=1.5in]{geometry}
\usepackage{parsetree}
\usepackage{verbatim}
\usepackage{subfig}
\usepackage{wrapfig}
\usepackage{tipa}
\usepackage{textcomp}

\begin{document}

%% Title ----------------------------------------------------------------------
\title{Proving Equational Haskell Properties Using ATPs}
\author{Dan Rosén}
\maketitle

%% Abstract -------------------------------------------------------------------

\newpage
\abstract{
In this work we prove a lot of nice equational Haskell properties by
a first order logic translation and using automated theorem provers.
The results are astounding, writing correct Haskell is nowadays a
breeze.
}

%% Acknowledgements -----------------------------------------------------------

\newpage
\pagestyle{empty}
\section*{Acknowledgements}
\vspace{5mm}
ACK, ACK, ACK, ACK, ACK, ACK, NAK

%% Table of Contents ----------------------------------------------------------

\newpage
\tableofcontents
\addtocontents{toc}{\protect\thispagestyle{empty}}

\newpage
\setcounter{page}{1}

\chapter{Introduction}

Induction and Haskell

\section{Outline}

Explain the outline of the report

\chapter{Background}

This work's main areas, Haskell, FOL, ATPs and some basic DT.

\section{Haskell}

Describe some of the features of Haskell, types, lazy evaluation,
bottoms, pattern-matching, data types, type classes

\section{First Order Logic}

Connectives, quantifiers, predicates, functions and constants.
Derivation rules and models.

\section{FOL and Automated Theorem Provers}

Use derivations or models to deduce absurdity.

\section{Domain Theory}

CPOs, monotonicity, continuity and admissible predicates.

\section{Related Work}

Isabel, Sledgehammer, Rippling and other techniques
(Productive Use of Failure)

Other proof assistants as Coq

Agda and dependent types, Agda and ATPs by Peter, Bove, Sicard
\cite{agdaatp} and Dependently Typed Programming Based on ATPs

Zeno

Corecursive techniques \cite{corecursive}, Hinze's papers, Generic
Approximation lemma, or have this in other chapters




\chapter{Technical Part}

Implementation technicalities.

\section{Translation to FOL}

Describe (and motivate here?) the intermediate language

Pattern-matching and bottoms

Higher order functions and function pointers

Axioms of disjointness

Axioms of projections and injectivity of constructors

Extensional equality and application of bottom

\section{Proof techinques}

An article:

\subsection{Definitional Equality}

Only on non-concrete types

Prove monad laws for the environment monad for instance

Mention (the unimplemented \verb;Prop (a -> b); - \verb;a -> Prop b; isomorphism and
its relation to seq and extensional equality)

\subsection{Structural induction}

Simle induction to more complex structural induction and its
implementation.

\subsection{Fixed point induction}

Describe how and why it works

Fixed point induction on several functions and subsets of the existing functions

\subsection{Approximation lemma}

After fixed point induction since it is an easy consequence of fixed
point induction

\chapter{Discussion}

\section{Results}

\section{Future work}

QuickSpec (IsaCosy)

Finite/Total domains and what is a terminating function, anyway?
Predicates, functions or fast and loose reasoning \cite{fastandloose}

Lemmas

Implications

Finite fixed point induction on terminating functions

\section{Conclusion}

%% References -----------------------------------------------------------------

\bibliographystyle{apalikeurl}
\bibliography{masterbib}
\end{document}
